Finding a killer technique... 

\subsection{SWOT Analysis}

Here follows our SWOT analysis
\subsubsection{Strengths}
\begin{itemize}
	\item We are all good PHP programmers, the language in which Moodle is written. 
	\item We all have a good understanding of DBMS and how to make a good database design. 
	\item We are some 14 people working together on this Moodle project which gives us a greater capacity of locating errors and problems before they happen. 
	\item We have a great cooperation with ELSA on AAU, who are the people in charge of deploying Moodle on AAU.
	\item We have a decent version control.
	\item We have an easily accessible platform.
\end{itemize}


\subsubsection{Weaknesses}
\begin{itemize}
	\item We do not have a product owner in this project, meaning there are no objective product owner or ``boss'' of this project. Therefore we have to make all decisions ourselves, which is not always ideal as we can have a somewhat colored view of the project.
	\item The members of our group have little to no knowledge of how Moodle is build and works ``behind the scene''. This means that we have to use a lot of time getting to know how Moodle works, instead of focusing on the actual assignments that we have.
	\item Nobody in the group have tried an agile development approach in such a large group before.
	\item We perform bad estimations, as we are not that experienced yet.
\end{itemize}


\subsubsection{Opportunities}
\begin{itemize}
    \item We have the opportunity to roll out our software to a large user group.    
\end{itemize}

\subsubsection{Threats}
\begin{itemize}
    \item A threat could be that since the user group is so diverse it could create problems with priorities.
    \item Other companies might launch a similar solution faster than us.
    \item Another threat could be that the university decides that Moodle is not worth spending time on.
    \item Project is to be continued by other people next year, these people could potentially ruin the project.
\end{itemize}

\subsection{Chosen Tools}
With a SWOT analysis in place it is time to look at how we can possibly fix some of our weaknesses. A weakness that is critical to fix or lessen is our lack of knowledge of Moodle. Therefore we will try some techniques that helps us fix this.
\begin{itemize}
    \item Modified Planning Poker - Two of our weaknesses are our limited knowledge of Moodle and somewhat poor estimations. With a modified version of planning poker we could get past both of these issues. It could be fixed by having a person from the outside, who is experienced, thus better at estimating, and knows Moodle quite well during Planning Poker. This outsider would be able to keep an objective look at the issues at hand and guide the group during estimations.
    \item Pair programming - We know that we are all good PHP programmers, and we also know that we all have little to no knowledge of the structure of Moodle and how we should implement our modules correctly. Therefore we find it a good idea to use Pair Programming. By using Pair Programming we have two developers working together on the same piece of code at the same time. Thereby we make sure to remove the worst bugs and implementation errors. We find this a good idea due to our lack of knowledge of Moodle's internal structure.
\end{itemize}

\subsection{Implementation plan}
Both of the tools are quite easy to implement. The modified planning poker only requires the external expert to participate, and we should be able to complete that and optimize the planning of the individual sprints. Pair Programming only requires us to work in pairs on one computer at a time, so we have two people writing the same program to remove as many errors as possible and combine our knowledge of Moodle.

