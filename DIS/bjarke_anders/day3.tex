
How it actually went...

\paragraph{}
As expected we encountered a lot of difficulties and challenges during our work with Moodle.
A lot of unexpected turn of events and surprises showed up during the development phase, forcing us to adapt our way of working with Moodle several times.
If we had chosen to go with traditional development, this would have been an issue as we would not have been able to adapt our process along the way. 
But because of our choice of an agile development method, we were able to continuously adapt our work and way of working with Moodle.

\paragraph{}
Based on our choice of an agile development method, we have evaluated what went well and what did not go as well. 

\subsection{What did not go so well}
Even though we thought we could use the interviewed people combined as a product owner, it did not go as expected because the product owner needs to prioritize the product backlog.
The interviewed people, however, did not have any experience nor the skills in order to prioritize this correctly. 
Therefore we ended up doing that ourselves, even though it is not an optimal solution when following the Scrum of Scrum development method.

\paragraph{}
Our aim to have frequent Scrum of Scrum meetings with one representative from each project group actually went fine. 
The problem with these meetings were that we rarely had attendance from Kurt or Saulius, and never both of them at the same time.
This is not an immediate issue, but it differentiated from our initial plans.

\paragraph{}
We aimed to have our Modified Planning Poker sessions with someone who knew their way around Moodle.
We were not able to find one such individual, however, so we had to base our individual estimations on our gut feeling and previous experiences. 
This was quite difficult, however, as none of us really knew anything about how Moodle was build or worked behind the scenes. 

\paragraph{}
Another aim was to conduct a lot of Pair Programming due to our lack of experience in working with Moodle.
This was not used as much as we expected, however, as we feel like we are all good programmers, and the problems we encountered in Moodle mostly were organizational or Moodle-framework related issues.
The actual programming-related issues were manageable by us individually, and so we were able to discuss a solution on the other Moodle issues. 
Therefore there were no real need for Pair Programming as such, so this idea became less relevant and thus barely used.

\subsection{What did go well}
The Scrum of Scrum meetings went incredibly well, we always knew exactly what the other teams were doing.
We ended up having bi-weekly meetings, or even more frequent during times where we had major issues with Moodle across all the smaller teams.
This meant that no team was stuck with issues they could not handle, but instead every team was working on solutions.

\paragraph{}
Even though we did not have an expert during our Planning Poker meetings we were still able to almost estimate correctly after a few iterations, as the amount of issues we did not foresee went down considerably.

\paragraph{}
Even though most people in the group had never worked with an agile development approach it was not a problem, and our agile development approach seemed intuitive to the whole team.

\paragraph{}
We also found that our daily morning stand up meetings worked really well. 
They gave all group members a good insight into the status of the project and what everybody else in the project group had done and were planning to do on any given day. 

\subsection{Conclusion}
The main issue we had was that we did not have a product owner.
This ended up hurting our prioritizing since it is difficult to prioritize as a programmer from a customer's perspective.

\paragraph{}
Even though we were not able to get a different product owner, and that some of the minor ideas could have been executed better we were able to pick out and follow an agile development approach that suited us.
The overall development went smooth and was successful.

\paragraph{}
The consequences this will have for our future development practice is that we will likely stick to scrum, and the agile development approach, unless the project is made specifically for the traditional approach.
The preferred approach will, however, be the agile approach.