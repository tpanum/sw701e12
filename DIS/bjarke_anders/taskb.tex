
\subsection{Project description}
The project on SW6 is quite different than other university projects. The entire team of students has been split up into two main groups, working on each their main project. Furthermore, we have been split up into small groups of four, each working on our part of the main project. That means that all the small groups in each main project will have to work together, unlike previous semesters where each small group worked individually. 

\paragraph{}
Our main group is working with E-learning and trying to improve and expand the Moodle platform by implementing PBL as a tool. This means that we have 4-5 small teams of up two four persons, each working on a subpart of the system, but who must also work together with the other groups. As a consequence of this, we have to make a joint decision on which development approach to use for all the groups within each main project of the semester. 


\subsection{Development Approach}
When deciding on a development approach we must first decide whether to use a traditional or an agile approach.
The traditional approach was discarded as it requires heavy documentation and is not ideal in situations where the project domain is rather unknown. The agile approach makes an unknown project domain less significant as new or changed requirements can be considered after each iteration, and the design of the software can be alternated to reflect these changes. Furthermore is the agile approach very suitable for the project we are working on as we will likely not finish the project within our time frame, and thus the project needs to be handed over to other students next year. This means we will be using an agile approach to develop.

\paragraph{}
Within the agile approach there are two very popular methods: XP and Scrum.

\paragraph{}
XP dictates that you do not document but instead let the code be the documentation. This means the requirements of the code are a lot higher, as it needs to very readable. It also means that XP is more suitable for programmers with lot of experience and expertise, which is not always the case with students. Furthermore this is not ideal as the project will most likely be handed over to other groups after this semester, and these groups will also consist of students. If new students have to take over next year it is essential with proper documentation. For these reasons we have decided not to go with XP.

\paragraph{}
Scrum, being the other popular choice within the agile approach, is suitable for small groups. We are a rather large group compared to the general size of scrum groups therefore a modified scrum method could be useful, e.g. Scrum of Scrums. Scrum of Scrum handles our situation rather well, as the main group is split into smaller groups (the project groups) and these smaller groups then use scrum. In Scrum of Scrums you have scrum meetings between the small groups, so everybody is aware what everybody is doing.
Scrum of Scrums is very suitable for our situation, but there are a few issues with Scrum that we have to consider before we can fully commit to using Scrum.
Scrum requires a Scrum Master, who is supposed to lead the Scrum meetings and facilitate the Scrum process, but otherwise not be a member of the development team. Since this is a student project, no one is going to have the required experience to act as a Scrum Master, and even if someone did, losing a developer would be unacceptable, instead we should make the Scrum Master a combined role of all the participants at meetings. Within Scrum there is the concept of a product backlog and a sprint backlog. The product backlog contains all the features we want to implement during the project and the sprint backlog contains the features we want to implement during the current sprint or iteration. Scrum includes a Product Owner, who is supposed to be the voice of the customer, and manage this product backlog. Again, since this is a student project we do not have anyone who fits this role, and thus we need to handle the product and sprint backlog ourselves. These backlogs could be created based on interviews with potential users of the project. The product backlog would be an ideal product to pass on to next years group, so they can continue where we left the project.

\paragraph{}
As we are working in small teams and considering the other circumstances under which this project will be executed, Scrum of Scrum seems as the most reasonable method to adapt. That being said, it does not fit our project exactly, so some modifications will be made in order to accommodate for the challenges described above.
After several master-meetings with all the small teams attending, we have decided to use an agile approach, and more specifically Scrum of Scrums.

\subsection{Implementation of Approach}
During the decision of using Scrum of Scrums we agreed on the following conditions:

\begin{itemize}
	\item We arrange frequent meetings where each team will be represented by a single team member. Kurt and Saulius, our supervisors, will also be attending. Joint decisions will be made here.
	\item Big group meetings where all the members of this project will meet. These meetings are with the purpose of sharing information and news on the project.
	\item Iterations and milestones. The actual development must be agile and in planned iterations. We will have common deadlines across the teams. 
	\item Everything must be documented and uploaded to our Mercurial (version control) server. We are also implied to comment our code in order to make documentation out of this at a later point.
\end{itemize}

We will be using a modified version of Scrum of Scrums as we have no real Product Owner the Product Owner will be all the interviewed people combined. Therefore we might not have a Product Owner to help with the designing phase during each iteration. The fact that we use a modified Scrum of Scrum implies we will be using a modified version of Scrum within the small teams as well. So instead of using a Product Owner we will be interviewing people before the initial designing process and these interviews will be the base for the product backlog. After each iteration we will hopefully be able to ask the interviewed further questions to maybe update the product backlog and create the sprint backlog.

The missing Scrum Master will be replaced by every participant at the meeting, and it will be the collective responsibility of everybody at the meeting to uphold these Scrum criteria.

\paragraph{}
These decisions mean we will be having daily scrum meetings within each small team, and weekly scrum meetings with a representative from each small group.
After each iteration we will have a meeting that everybody attends to share information and news on the project.
As the project groups will be using Scrum there should be a focus on testing before the integration test after each iteration. Documentation tools should furthermore be used throughout each iteration to document the implementation.
The small teams have a common deadline for each iteration and the project will have integration of all the subsystems into the main system after each iteration.