\section{UML notation} \label{section:uml_notation}

The UML diagrams in this report will be guided by the UML standard guide written in cooperation by several different software companies \citep{UML_notation}.
Because this project uses Active Relations (as a consequence of our choice to use Rails) it is possible to model the database as a class diagram \citep{active_records}. 
In other words the classes in the system will represent the database attributes with their own attributes. \\

The UML notation in this report:
\begin{itemize}
	\item PK attribute - means that the attribute is a primary key.
	\item FK attribute - means that the attribute is a foreign key.
	\item attribute : type - all attributes will have a type e.g. \verb+id : int+.
\end{itemize}

For an example see Figure~\ref{fig:UML_class_diagram}