
% Hvordan er web applikationen opbygget?
%	Rails framework
%	MySQL database
% Hvilke gems bruger vi, og hvordan bruges de?
%	mysql2
%	daemons-rails
%	eventmachine
%	sht_rails
%	swfobject-rails
% Spændende ting
%	CRUD
%	Integration af streaming i web-applikationen
%	Privileges

\section{Web-interface}
The web-interface is implemented using Rails as described in Section~\ref{sec:programming_language}.
This framework enforces a number of standards that is followed in the implementation of the web-interface.
The most significant is CRUD, mentioned in Section~\ref{sec:master}.
The web-interface is implemented using the design mentioned in Section~\ref{sec:master} and Appendix~\ref{app:controller_actions}.
A number of Rails-specific tools, named \emph{Gems} \citep{Rails_Gems}, which are plugins for Rails written by the community, provides us with a number of tools that we can use and build our own application on. 
Five different Gems are used in \projectname{}. 
These are: \\

\begin{itemize}
	\item mysql2
	\item daemons-rails
	\item eventmachine
	\item sht\_rails
	\item swfobject-rails
\end{itemize}

\projectname{} uses a relational MySQL database to store all informations in.
Ruby and Rails is not able to connect with MySQL out of the box. 
Therefore the \emph{mysql2} \citep{Rails_mysql2} gem is is used, which provides a simple interface for Ruby to connect with a MySQL database.


