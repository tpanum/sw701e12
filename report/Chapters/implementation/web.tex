
% Hvordan er web applikationen opbygget?
%	Rails framework
%	MySQL database
% Hvilke gems bruger vi, og hvordan bruges de?
%	mysql2
%	daemons-rails
%	eventmachine
%	sht_rails
%	swfobject-rails
% Spændende ting
%	CRUD
%	Integration af streaming i web-applikationen
%	Privileges

\section{Web-interface}
The web-interface is implemented using Rails as described in Section~\ref{sec:programming_language}.
This framework enforces a number of standards that is followed in the implementation of the web-interface.
The most significant is CRUD, mentioned in Section~\ref{sec:master}.
The web-interface is implemented using the design mentioned in Section~\ref{sec:master} and Appendix~\ref{app:controller_actions}.

%\subsection{Gems for Ruby on Rails}
A number of Rails-specific tools, named \emph{Gems} \citep{Rails_Gems}, which are plugins for Rails written by the community, provides us with a number of tools that we can use and build our own application on. 
Five different Gems are used in \projectname{}. 
These are: \\

\begin{itemize}
	\item mysql2
	\item daemons-rails
	\item eventmachine
	\item sht\_rails
	\item swfobject-rails
\end{itemize}

\projectname{} uses a relational MySQL database to store all informations in.
Ruby and Rails is not able to connect with MySQL out of the box. 
Therefore the \emph{mysql2} \citep{Rails_mysql2} gem is is used, which provides a simple interface for Ruby to connect with a MySQL database.

The \emph{daemons-rails}�\citep{Rails_daemons_rails} is used for the daemon running in \deno{M}. 
It accepts incoming requests from \deno{S}'s and uses the models in the web-application to interact with these.
To make sure that the daemon can use these, the daemons-rails gem is used.

The \emph{eventmachine} Gem \citep{Rails_eventmachine} is used to allow our web-application to easily interact with other programs using TCP/IP.
The daemon running on \deno{M} also uses this to communicate with \deno{S}'s when exchanging informations regarding session keys. 
Whenever a user requests access to a drone, the request of a session key is saved in the database. 
The daemon running on \deno{M} takes this request and uses the eventmachine to connect to the \deno{S} associated with the drone that the user wishes to interact with, in order to sync session keys and verify that the connection between the user and drone is valid. 

The \emph{SHT\_Rails} Gem \emph{Shared Handlebars Template for Rails} \citep{Rails_sht_rails} is used in the view of the web-interface and enables the application to render a view both as HTML and JSON, making it easier to integrate AJAX functionalities.
This is used in the view, where AJAX is implemented as described in Section~\ref{sec:browser_technologies}

The \emph{SWFObject} Gem \citep{Rails_swfobject} allows us to easily display a swf flash object in the view of \projectname{}, which is used when a user is interacting with a drone by either viewing its video feed or sending new commands to it. \\

All of the listed Gems are plugins written by third party that provides some functionality that was need in \projectname{}.

