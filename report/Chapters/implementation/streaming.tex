\section{Streaming}
\subsection{Gstreamer}
\label{sec:gstreamer}
As described in Section~\ref{sec:tools_streaming} the tool Gstreamer is used for reading the drones� video feed and forwarding it to CRTMP.
It is a pipeline based tool, and can be run through the command line using the \verb+gst-launch-0.10+ command.
A pipeline in gstreamer is a set of plugins the video feed is processed by seperated by a \verb+|+.
The first plugin in the pipeline is called a source element, and the final element is a sink.
The plugins between the source and sink element consist of a set of video and audio processing plugins, such as decoders and encoders.

>>FIGURE af pipeline princippet<<

The source element is capable of receiving a video feed via a specific protocol from a defined destination.
As described in Appendix~\ref{app:ar_drone_specification} the drone outputs its video feed using TCP on port 5555, and its IP on its network .
To read the stream the Gstreamer plugin \verb+tcpclientsrc+ is used.
It has a \verb+host+ and a \verb+port+ as parameters and the first part of the pipeline is: \verb+gst-launch-0.10 tcpclientsrc host=192.168.1.1 port=5555+.\\

The data available for the next plugin is a H.264 encoded video frame with PaVE headers.
The PaVE headers cannot be interpreted by Flash, and are therefore parsed by gstreamer using the \textit{paveparse} plugin\citep{paveparse}.
Paveparse removes the PaVE headers and send the remaining H.264 data down the pipeline.
Furthermore it handles the delay that comes from the drone using TCP to stream its video feed, by ignoring lost frames, and always displaying the newest received frames.\fxfatal{weak s�tning}.

As specified in Section\fxfatal{skriv det og ref til section tools} the video output of gstreamer should be in FLV.
Two pipelines can be made capable of this..
One is using a H.264 decoder to create raw video data, and then encode it as FLV.
The other is using a FLVmuxer which uses a workaround to convert the H.264 video data to FLV.
The best option is to use the H.264 decoder, however, as the H.264 data has no headers, due to \textit{paveparse}, the FLV encoder cannot properly encode the video data after it has been decoded.
Therefore an FLVmuxer is used to create FLV output.

With a FLV the sink element can be added to the pipeline.
As the video feed is forwarded the RTMP server CRTMP, the sink element is \verb+rtmpsink+.
It has one parameter called \verb+location+, which is the url of the RTMP server with an extension specifying the url of the stream on the RTMP server\fxfatal{D�rlig s�tning}.
As CRTMP runs on the same physical machine as Gstreamer the rtmpsink looks as follows:
\verb+rtmpsink location='rtmp://0.0.0.0/live/myStream'+.

The pipeline to forwarrd the drones� video feed is therefore: \verb+tcpclientsrc host=192.168.1.1 port=5555 ! paveparse ! flvmux name=mux ! rtmpsink location='rtmp://0.0.0.0/live/myStream'+.

There is however the chance some plugins process data faster than others, causing data to pile up at certain plugins if it is send through unrestricted.
This can be solved using the plugin \verb+queue+.
\verb+queue+ is a data queue that queues data until e.g. the queue reaches a specified size.
The queue element is added between each plugin except \verb+tcpclientsrc+ and \verb+paveparse+, as there is no need to queue between the entry point and the first plugin.
The resulting pipeline is therefore: \verb+tcpclientsrc host=192.168.1.1 port=5555 ! paveparse ! queue ! flvmux name=mux ! queue ! rtmpsink location='rtmp://0.0.0.0/live/myStream' --gst-plugin-path=.+.

gst-launch-0.10 -v -m testvideosrc ! queue ! x264enc pass=pass1 threads=0 bitrate=1536 tune=zerolatency ! queue ! flvmux ! queue ! rtmpsink location='rtmp://0.0.0.0/live/myStream'


sudo gst-launch-0.10 -v -m tcpclientsrc host=192.168.1.1 port=5555 ! paveparse ! queue ! flvmux name=mux ! queue ! rtmpsink location='rtmp://0.0.0.0/live/myStream' --gst-plugin-path=.

\subsection{C++ RTMP Server}
\label{sec:CRTMP}
As described in \ref{sec:tools_streaming} C++ RTMP Server is used as the server tool between Gstreamer and the flash application.

CRTMP is found at http://www.rtmpd.com and comes in a source package ready to run.
It can be runned with the linux command \verb+sh+, which is the bourne-shell or a compatible shell.

There are four possibilities:

\begin{itemize}
   \item run_all.sh
   \item run_all_deamon.sh
   \item run_flvplayback.sh
   \item run_flvplayback_deamon.sh
\end{itemize}

If the command \verb+sh run_all.sh+ CRTMP will use the config file \verb+all.lua+ and run with this configuration.

The flvplayback was used, because this is the flash streamer application, and the configuration file \verb+flvplayback.lua+ was configured like seen in listing~\ref{lst:flvplayback}.

\begin{lstlisting}[style=sourceCode, language=C, caption=flvplayback.lua configuration, label=lst:flvplayback]
acceptors =
                       {
                               {
                                       ip="0.0.0.0",
                                       port=1935,
                                       protocol="inboundRtmp"
                               }
                       },
                       externalStreams =
                       {
                               {
                                       uri="rtmp://flash.oit.duke.edu/vod/MP4:test/brunswick.m4v",
                                       localStreamName="test",
                                       forceTcp=true
                               }
                       },
\end{lstlisting}


It is important to understand that \verb+acceptors+ is inbound and \verb+externalStreams+ is outbound.
Inbound is which ports CRTMP will let users connect to and \verb+externalStreams+ is where the stream is coming from.
As seen in listing~\ref{listing:flvplayback} there is setup a test source which is a VoD, Video on Demand.
This test source was used to test if CRTMP was running correctly.
To logon to the stream Video Lan Client, VLC, was used because it have the same capabilities as flash when it comes to streaming RTMP.

A connection is established with an URL of the format \verb +rtmp://[server address]/[application]/[streamName]+, where the \verb+server address+ is the address of the CRTMP server, the \verb+application+ is the application where CRTMP lookup e.g. live or media, and the \verb+streamName+ is the name of the stream.

When a user connects to CRTMP and the \verb+application+ is live CRTMP will try to find a live stream that correspond to the link name.
If no such live streams is found CRTMP will look in the media folder and try to find file streams.
If no file stream is found CRTMP are going to wait for the live stream \verb+streamName+.

C++ RTMP Server also uses RSA keys.
It is possible to make users and add these users to realms. This can be used as an extra layer of security but is not used in \projectname{}.

\section{Output stream}
Following the setup of Gstreamer and CRTMP the next step is reading the output stream and displaying it in a Flash application.
Due to the PaVE headers this is however not possible.

\subsection{Issues with PaVE Headers}
The documentation for the PaVE headers can be found at \citep[page. 59-60]{ardrone_developer_guide}.
As described in section~\ref{sec:gstreamer} Flash applications are unable to interpret the PaVE headers, and as a result unable to display a video stream encoded with them.
Therefore they must be removed or replaced with another header for the stream to be readable.
The implemented solution removes them with the \verb+paveparse+ plugin, described in section~\ref{sec:gstreamer}.
Removing the header leaves raw video data, with no information about how to interpret it.
As a result the Flash application, VLC, and other video players are unable to interpret the video data.

The PaVE headers result in a situation that makes displaying the video stream in Flash impossible, as the video feed cannot be interpreted by a Flash application both with and without them.

\subsection{Test input}

In Gstreamer there is a test source which is called \verb+videotestsrc+. This creates a test video stream as seen in figure~\ref{fig:videotestsrc}.

If this source is used instead of the one from the drone it is possible to show that this solution will work.

This solution can be seen in figure~\ref{fig:working_solution}