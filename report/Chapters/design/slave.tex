\section{Slave}
\label{sec:design_slave}
The functionality of Slave is as follows, derived from Section~\ref{sec:communication_network} and Section~\ref{sec:functionality_distribution}.

\begin{enumerate}
	\item Establish and maintain a wireless connection to the drone.\label{enum:wireless}
	\item Send initialization message to M when powered on, as seen in Figure~\ref{fig:sequence_diagram}.\label{enum:initialization}
	\item Create and send session keys upon request.\label{enum:session}
	\item Receive flight-operations from \deno{B} and send them to the drone.\label{enum:actions}
	\item Read video feed from the drone and forward it to \deno{B}.\label{enum:stream}
\end{enumerate}

The slave is connected to the drone and the Internet. %\fxfatal{Hvilket ord skal bruges her? - det der står? Det er jo rigtigt /a}.
Therefore two network interfaces are required, one of which must be a wireless network interface (WIFI-interface), as all the communication with the drone is wireless as described in Appendix~\ref{app:ar_drone_specification}.
\deno{S} will automatically attempt to establish a connection to the drones wireless network when powered on.
Following this the initialization message is send to \deno{M}.
When the initialization is succesfully completed, \deno{S} is ready to use.
Each \deno{S} will have a unique identifier called serial code. \\

The initialization message is send from \deno{S} to \deno{M} as JSON, see Section~\ref{sec:tools}.
It will contain \deno{S}' serial code.
The initialization message is send to \deno{M} when \deno{S} is powered on.

A session key is generated on \deno{S} when a request is received, see item~\ref{enum:session}.
It is not be possible to communicate with \deno{S} without a valid session key.
The session key is generated on \deno{S} to ease the load on \deno{M}.
As mentioned in Section~\ref{sec:communication_network} there can only be one active session key active on \deno{S}.
A session key is a 40 character string consisting of letters and numbers.
This was chosen based on \citep{password_length} which states that length is stronger than complexity when creating passwords.
A session key of length 40 was deemed sufficient.
Following the generation of a session key it is stored on \deno{S} and send to \deno{M}.
While a session key is active on \deno{S} only messages containing it are forwarded to \deno{S}' associated drone.
The session key is send to \deno{M} as JSON, see Section~\ref{sec:tools}.
Session keys will be deleted from \deno{S} if the connection times out.

\deno{S} receives flight-operations from \deno{B} which must be forwarded to the drone, see item~\ref{enum:actions}.
The flight-operations must be read and converted to an \verb+AT command+ for it to be interpretable by the drone, see Appendix~\ref{app:ar_drone_specification}.

Item~\ref{enum:stream} will be explained in Section~\ref{sec:tools}, as already existing technologies will be used.

%The need for session keys was described in Section~\ref{sec:communication_network}.
%When a client tries to access the drone attached to a give slave, the clients connection must be verified using these session keys.
%So the slave has to make these and synchronize with \deno{M}.

%As soon as \deno{S} boots, it should connect to the WIFI broadcast by \deno{D}.
%\deno{S} will come with some pre-installed software on deployment.
%A part of this will be instructing the operating system, that when it is booted and the drivers for the wireless network interface are loaded, it must connect to a %defined network.
%This network will be defined to be the one broadcast by the drone that \deno{S} is parred with. \\

%The four remaining areas of responsibility for \deno{S} requires some custom program to be executed.
%Connecting to \deno{M} when connected and creating and sending session keys are designed as individual scripts that are executed on boot and on request.
%The dissemination of flight commands to the drone and steaming of its video feed is more interesting, however. \\

%As described in Appendix~\ref{app:ar_drone_specification} commands are send to \deno{D} as UDP packages, and that \deno{S} must send as many as 30 packages per second.
%Furthermore \deno{D} will return to its default hovering position, if it does not receive any packages.
%To send a sufficient ammount of packages requires a state-full connection, as the delay with a state-less connection is to large to properly control the drone\fxfatal{statement: soirve}.
%Therefore we have to create a daemon running on \deno{S} which accepts packages, formats them to the correct format and sends them on to \deno{D}.  \\

%The video feed from \deno{D} that the client must be able to see communicates the other way.
%It is recorded by \deno{D}, sent to \deno{S} and from there it must be sent to the client.
%It is out of the scope of this project that we design an implement our own streaming protocol.
%Therefore an existing streaming protocol will be used to stream the video feed from \deno{D} to the client.
%This is described in further details in Section~\ref{sec:tools_streaming}.




% Skriv at vi bruger eksisterende steaming tools fremfor at lave et streaming tool selv, da det er udenfor projektets scope
