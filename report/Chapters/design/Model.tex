\section{Model}
In this section the model of \projectname{} will be presented and explained.
First the objects that needs to be represented by the model are identified, and then their relationships will be described.

\subsection{Objects}
\label{subsec:objects}
\fixme{This might fit better in the analysis}
The core aspects that must be represented is users and drones.
A users' access to the system will need to be restricted.
This restriction will be represented through privileges that grant a user access to system functionalities.
Granting privileges directly to a user is not the ideal case in most situations.
The system will contain global privileges all users need access to.
Assigning global privileges to each user individually is not an ideal solution.
Therefore this will be handled via roles.
A role will in \projectname{} be a set of privileges and assignable to multiple users.

A drone is controlled by sending a set of ordered instructions\fixme{ref to section when it is written} to the drone.
A log of flights made by the drone will need to be kept, to be able to recreate previous flights, and examine a flight that might have caused the drone to crash.
This log will be referenced in the system as a flight plan.
A flight plan will consist of a set of actions taken by the drone, each referencing a specific instruction send to the drone.

A structured list of the object can be seen in Figure~\ref{tab:model_objects}.

\begin{figure}[H]
\begin{center}
\begin{tabular}{ |c| }
  \hline              
	\textbf{Object}\\ \hline
	User \\ \hline
	Role \\ \hline
	Privilege \\ \hline
	Drone \\ \hline
	Flight plan \\ \hline
	Action \\ \hline
	Instruction\\ \hline  
\end{tabular}

\caption{Use Cases}
\label{tab:model_objects}
\end{center}
\end{figure}

\subsection{Relationships}
The objects described in Section~\ref{subsec:objects} are connected through a set relationships.
The relationships will be of the form

\subsection{Privileges}
A user is granted privileges through a whitelist.
Every privilege not found in a users whitelist is implicitly blacklisted.
As described in Section~\ref{subsec:objects} a user can be granted privileges through roles.
There are however a set of cases where handling privileges only through roles is not ideal.
As an example a user \verb+u+ might need to have his access to a specific privilege \verb+p+ in a role \verb+r+, he is a member of, revoked.
A possible solution is to remove the user from \verb+r+ and add him a to new role \verb+r1+ that contains all privileges in \verb+r+ except \verb+p+.
The problem with this solution arises if a new privilege \verb+p1+ is to be added to \verb+r+.
The user \verb+u+ is natively a member of \verb+r+, but has the role \verb+r1+ to exclude him from the privilege \verb+p+.
To grant \verb+u+ \verb+p1+, \verb+p1+ must be added to \verb+r1+ as well as \verb+r+.
Granting privileges only through roles can therefore quickly result in a lot of unstructured data that needs to be maintained.
To solve privileges need to be manageable on the user level.


%As an example a user might need a specific privilege that allows him to fly a specific drone.
%Adding the user to a group \verb+p+ with that privilege would not work as he would then additionally receive every other privilege in \verb+p+.
%The problem can be solved by giving each user a personal role containing all their uniquely assigned privileges.