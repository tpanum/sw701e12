\section{Development Method}
We have had experience with different development methods from previous semesters, more specifically Scrum and XP.
During these semesters we learned to take advantage of using common Scrum principles such as stand-up meetings, making short and precise iterations and to postpone tasks to the next iteration if not completed.
We are, however, aware that Scrum might not fit exactly on this project.
Therefore we have analyzed what is required from this project and how we need to modify Scrum in order to fit to our needs. \\

We will implement all the principles from Scrum. In addition to this, we have changed and/or added the following to our modified version of Scrum:
\begin{itemize}
	\item Pair programming. We have decided to add the principles of pair programming, found in XP, to our project. We are working in an area of great uncertainty, as we have no prior experience with a lot of the problem domain. To lower the number of errors and faster get an understanding of our system, we have decided to implement Pair Programming.
	\item Refactoring. Also due to the uncertainty of this project, we have added refactoring sessions to our iterations. These are also normally found in XP. They are added to enable us to improve our code over time.
\end{itemize}

The reason why we have chosen not to use any of the traditional development methods such as the Waterfall model is that its procedural and locked procedures does not fit a project, where the problem domain contains a lot of uncertainties and unknown areas. We need to be able to look back at what we have done and learned in the previous iterations, refactor, re-design, improve and re-implement parts of the solution. Our modified version of SCRUM allows us to accomplish this. 

