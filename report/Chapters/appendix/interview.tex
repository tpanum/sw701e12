\chapter{Interview with Lytzen IT}
The interview were setup with Lytzen IT via email and they were the only alarm company to contact us back.

Lytzen IT were presented by both S�ren Ole S�ndergaard and Jesper Toft. S�ren is one of the partners that owns Lytzen IT and Jesper is their alarm and security specialist.
The meeting found place at Lytzen AS and Lytzen IT's headquarters in Hj�rring, Friday the 12 December.

In the meeting the project were presented and the idea about drone surveillance. What the thought was and what the system could solve of problems in todays world. 
An example was surveillance of AAU buildings where there are install contacts on every window in this way they can tricker an alarm. The alarm will only go off in a certain time interval of the day if it is a workday and all the time if it is in the weekends. One of the problems is if a student accidently opens a window and the alarm goes of a guard must come and check it out which is very costly. Here the drone could be placed inside the building and flown around to check alarms before sending a guard.

Lytzen IT had some cases where they either have used a lot of small cameras or one big dome camera to make a CCTV, Closed-Circuit TeleVision. A dome camera is a camera which can turn 360$^\circ$ and in the big models have a great optic zoom. Because of all these features these dome cameras can surveillance a big area but there is some short comings. They are expensive, they have to be placed high up to be able to see everything and if there are objects they cant see ``behind'' them. For out the cameras being very expensive the cables for these cameras are even more so. They need to be shielded against weather and being in the ground. They need to be laid in a certain depth.
In large outdoor areas the drones are probably more suited than the stationary cameras. They can go almost everywhere and are not as expensive as the big dome camera or a lot of small cameras with all the cables needed. But the drone system presented also have some short comings. Firstly it is using the global Internet. They think as CCTV is a closed-circuit our drone stream and communication should be the same.
Lytzen also pointed out that the station where the drone needs to charge needs to have some technology that ensures the drone can just land and then charge without any human interaction. The ``charge to fly time ratio'' should be better before it can be used in a real case.
Another thing was that the drone maybe should use radio communication instead of WIFI because it would both give a more secure line and make the range of the drone higher. Also they think that the drone should be automated more so it is not dependent on a pilot all the time. But they still thought that it will be the future in 5 to 10 years. Surveillance is about getting evidence for the police, insurance company and keeping the evidence safe. The drone can also have a intimidating effect. This is why they think that drone surveillance is the future.
The last problem that Lytzen pointed out was the law. In Denmark you may not record any public places. This could be troublesome when using a drone. But Lytzen IT thinks this is likely to change as Denmark is moving against a total surveillance society. 