\chapter{AR Drone Technical Specification}
%In this section the technical description of the AR Drone 2.0 Parrot will be presented as it influences the design of \projectname{}.
%The source for this section is the AR Drone 2.0 Parrot SDK.\fxfatal{ref til source n�r den er i bibtex}

The AR Drone has its own wireless network that one must connect to in order to control it.
It will always have the ip \verb+192.168.1.1+.
Communication with the drone is done using this IP through three ports:
\begin{itemize}
	\item 5554: The drone sends out information on this port to its clients. This information is e.g. its status(\textit{flying}, \textit{landed} etc.), and its speed. The information send from the drone is reffered to as \textit{navdata}.
	\item 5555: The drone sends out a video stream on this port via TCP.
	\item 5556: The drone is controlled and configured by sending \textit{AT commands} to it on this port via UDP.
\end{itemize}
 
AT commands are text strings, which refer to a specific function call in the drones library, as such parameters are send to the drone along with the command.
The most important AT commands are:
\begin{itemize}
	\item \verb+AT*FTRIM+ - configures the drones notion of a horizontal plane. Drone must be on ground when the command is send.
	\item \verb+AT*REF+ - Take off, land, and emergency stop command. Takes an integer value as argument.
	\item \verb+AT*PCMD+ - Controls the drone. Takes 5 arguments: a flag, and four integer values which defines the drones movement. When a \verb+PCMD+ command is received the drone sets its control values to those received and then resets them. For consistent movements the drone must receive at least 30 packets per second. 
\end{itemize}

The video stream send by the drone on port 5555 is encoded in the H.264 format.
Network video streaming is done by sending the video frames individually.
Each video frame has a custom header containing metadata on the frame.
The stream send by the drone uses a unique header PaVE \textit{(Parrot Video Encapsulation}.