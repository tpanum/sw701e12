\section{Use cases}
The functionality of the system will be defined through use cases, describing how a user of the system would desire to interact with the system.
The use cases are derived from the requirements described in Section~\ref{sec:problem_domain}.
The use cases can be seen in Table~\ref{tab:use_cases}.\\

\begin{figure}[htb]
\begin{center}
\begin{tabular}{ | l | p{6cm} | }
  \hline              
	\textbf{ID} & \textbf{User Case} \\ \hline
	%%%%%%%%%%%      %%%%%%%%%%%S
	%%%%%%%%%%% OBS! %%%%%%%%%%%
	%%%%%%%%%%%      %%%%%%%%%%%
	% Inden tallene �ndres, skal der lige gøres opmærksom på det, da de bruges andre steder.
	1 & As a user I want to be able to login into the system.\\ \hline
	2 & As a user, when I am logged in, I want content shown based on my privileges. \\ \hline
	3 & As a user I want to pilot a specific drone.\\ \hline
	4 & As a user I want to view the video feed of a specific drone. \\ \hline
	5 & As a user I want to be able to grant or revoke other users the privileges that I am an admin of. \\ \hline
	6 & As a user with rights I want to change the settings for a specific drone. \\ \hline
	7 & As a user I want to be able to add and remove a drone to a company. \\ \hline
	8 & As a user I want an overview of all drones my privileges grant me access to.\\ \hline
	9 & As a user I want to be able to log out.  \\ \hline
	10 & As a user with rights I want to be able to create and edit users.  \\ \hline
	11 & As a user with rights I want to be able to create a new company.  \\ \hline
	12 & As an owner of a company, I want to be able to edit the company.  \\ \hline
	13 & As a user I want to be able to edit privileges, that I am allowed to edit, for another user.  \\ \hline
	14 & As a user I want to be able to become a customer.  \\ \hline
	15 & As a user I want to be able to add and remove privileges from a role.  \\ \hline
	16 & As a user I want to be able to add roles to users within the company I am allowed to do so.  \\ 
  \hline  
\end{tabular}
\caption{Use Cases}
\label{tab:use_cases}
\end{center}
\end{figure}

The use cases define a set of objects which must be present in \projectname{} in order for the use cases to be implementable.
The objects reflect elements in the problem domain which must be modelled in the system.
The objects are \textit{users}, \textit{drones}, \textit{companies}, \textit{roles}, and \textit{privileges}.
Everybody using \projectname{} are classified as \textit{users}, including customers, owners of companies, system admins etc.
However as reflected in the use cases \textit{users} will not have unrestricted access to the system.
Access to the system will be restricted through \textit{privileges}.
As defined in use case \#8 privileges grants access to functionality, meaning a \textit{user} will not have access to anything by default.
A \textit{drone} refer to a physical drone.
\textit{Companies} is used as a grouping of associated \textit{users}, \textit{drones}, and \textit{roles}.
As an example a company might purchase a set of drones for surveilling their property.
The company would need a set users for its employees, and a set of privileges granting access to its drones.
The \textit{company} object would handle this grouping.
Use case \#15 defines a role as at set of privileges, that can be associated with a \textit{company} and granted to \textit{users}.\\

As video surveillance is concerned with sensitive information, security in \projectname{} is important.
This is reflected in use case \#1 and \#2.
User authentication is used as the user must login to access to the content of system.
Access is then further restricted with \textit{privileges}, as the user's access to content is based on his privileges.
Privileges must therefore be designed to be applicable to all aspect of \projectname{}, and be able to restrict access to all parts of the system.
Restricting users access to certain parts of the system once they have gained entry is however not sufficient security for such a system.
It must also be considered how the system can be made secure from external attacks.


