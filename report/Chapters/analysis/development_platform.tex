\section{Development Platform}
At this point we know that we will need a variety of different functions and methods to make the systems as described.
The problem domain implies that we need both web development and server-side scripting.
This means that we will have to make two decisions: First the operating system of the server and secondly the programming language in which our program will be written.

\subsection{Server operating system}
There are two major operating system families to choose from, which each has some potential operating systems beneath them:

\begin{itemize}
	\item Windows Server family
		\begin{itemize}
			\item Windows Server 2003
			\item Windows Server 2008
			\item Windows Server 2012
		\end{itemize}
	\item UNIX family
		\begin{itemize}
			\item Linux
				\begin{itemize}
					\item Debian/Ubuntu
					\item Fedora
					\item CentOS/RedHat
					\item Various other distributions of Linux
				\end{itemize}
			\item BSD
			\item Mac OSX Server
		\end{itemize}
\end{itemize}

We have chosen to use Linux Debian due to the following reasons:

\begin{itemize}
	\item We find that the various Linux distributions are easier to use for project like these, where various web applications and server-side scripts are executed on the server. 
	\item It is easy to get started on the project, as installation of the programs needed is simple.
	\item Linux Debian is open-source and does not require a license to install or use. 
	\item Linux Debian is one of the most well-seen operating systems amongst system administrators, getting credit for its maintainability, size, customisability and community support.\citep{why_debian}
\end{itemize}

It is important to underline that any of the listed operating systems are fully capable of handling any task we might ask of it in this project. 
Deciding which operating system to use is often considered a ``religious'' choice, as different system administrates has each there preferences.
Our preference is Linux Debian.

\subsection{Programming language}
Our choice of operating system limits us in which programming languages that we can use. 
Programming languages such as .NET that are not natively written to work on other platforms than Windows Server are discarded as we do not consider emulating tools such as Wine a sustainable solution. 

We have some requirements to the programming language that we want to use.
These are:

\begin{itemize}
	\item It must support object oriented programming
	\item It must have native MySQL libraries
	\item It must be open-source and have an active community to ensure continuous bug-fixing etc. or be a corporate language that is professionally maintained.
\end{itemize} 

This leaves us with a number of options that we have narrowed down to the ones we already have little to some knowledge about. 
These are:
\begin{itemize}
	\item Java
	\item Python
	\item Ruby
	\item Perl
	\item PHP
\end{itemize}

Just as when you are choosing which operating system to use, choosing which programming language to use can be considered a bit ``religious''.
All of the above are semantic equivalent and runs on Linux Debian, so either would work with this project.
However, we have chosen to go with Ruby on this project. 
The reasons why are:

 
We have decided to use the programming language Ruby and the web development framework Ruby on Rails as this gives us a number of advantages:

\begin{itemize}
	\item Ruby itself is easy to install server-side.
	\item Ruby works well as a server-side scripting language.
	\item Rails provide an easy-to-use framework for web-development with Ruby.
\end{itemize}

Ruby is a object oriented scripting language.
Besides working as a server-side scripting language it also works for web programming as several different web servers - e.g. Apache - has modules to support Ruby.

Ruby on Rails \emph{ROR} is a open-source web framework for Ruby that is optimized to make programming as easy and efficient as possible.
By using this framework on the web application side of the project, we give ourselves and the project an advantage as this provides a lot of built-in features, such as support for the MVC model, ActiveRecords etc. that enables us to be more efficient and write better quality code.
ROR also has build in features that make it easier to be multiple programmers working on the same project.
As an example we can look at ``Migrations'', which is ROR' way of handling databases.
Instead of having one big database that each developer connects to, to make sure that everybody is always working on an updated database, ROR uses these migrations to save changes to the databases.
When a developer pulls the newest revision from the repository, he is able to run a command to run migrations.
This will initialize a part of ROR that updates the developers local database and enables him to keep working.
If he makes some changes to the database, he simply makes a new migration document, pushes it to the repository and the rest of the team can follow the same procedure.

This is one of the features of ROR that helped us to make the decision to use it. \\

Keeping in mind that this is a student project, Ruby is also a natural choice as we in the group have little to no knowledge of how Ruby works.
This is a great opportunity for us to lean how to use another programming language. \\

We also know that we will be communicating long distance with the drone.
We define ``Long Distance'' in this context as communication between two computer devices that are not on the same LAN. 
This will require some sort of server or gateway between the web application and the drone.
We also already know that we have to build some web application and server side scripting which both requires some soft of hosting.




%We have decided to use Linux Debian for this purpose, as Linux provides a lot of useful and built-in features for hosting both web applications, server side scripting and databases.
