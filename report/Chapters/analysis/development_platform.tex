\section{Development Platform}
At this point we know that we will need a varity of different functions and methods to make the systems as descriped.
The problem domain implies that we need both web development and server-side scripting. 
Therefore we have decided to use the programming language Ruby and the web development framework Ruby on Rails as this gives us a number of advantages:

\begin{itemize}
	\item Ruby itself is easy to install server-side.
	\item Ruby works well as a server-side scripting language.
	\item Rails provide an easy-to-use framework for web-development with Ruby.
\end{itemize}

Ruby is a object oriented scripting language.
Besides working as a server-side scripting language it also works for web programming as several different webservers - e.g. Apache - has modules to support Ruby. 

Ruby on Rails is a open-source web framework for Ruby that's optimized to make programming as easy and efficient as possible. 
By using this framework on the web application side of the project, we give ourselves and the project an advantage as this provides a lot of build-in features, such as support for the MVC model, ActiveRecords etc. that enables us to be more efficient and write better quality code.
Ruby on Rails also has build in features that makes it easier to be a team of programmers working on the same project.
As an example we can look at ``Migrations'', which is Ruby on Rails' way of handling databases.
Instead of having one big database that each developer connects to, to make sure that everybody always is working on an udpated database, Ruby on Rails uses these migrations to save changes to the databases.
When a developer pulls the newst revision from the repository, he is able to run a command to run migrations.
This will initialize a part of Ruby on Rails that updates the developers local database and enables him to keep working thereforth and on. 
If he himself makes some changes to the database, he simply makes a new migration document, pushes it to the repository and the rest of the team can follow the same procedure. 

This is one of the features of Ruby on Rails (which we will go into details with later) that that helped us to make the decition to use it. \\

Keeping in mind that this is a student project, Ruby is also a natural choise as we in the group have little to no knowledge of how Ruby works. 
This is a great oppotunity for us to lean how to use another programming language. \\

We also know that we will be communicating long distance with the drone. 
This will require some sort of server or gateway between the web application and the drone.
We also already know that we have to build some web application and server side scripting which both requires some soft of hosting.
We have decided to use Linux Debian for this purpose, as Linux provides a lot of useful and built-in features for hosting both web applications, server side scripting and databases. 
