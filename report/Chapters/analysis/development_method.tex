\section{Development Method}\label{sec:developmet_method}
%P� hvilken baggrund v�lges en udviklingsmetode
%Under hvilke omst�ndigheder udf�res dette projekt
%Hvilke udviklingsmetoder er brugbare baseret p� omst�ndighederne
%Hvilken udviklingsmetode bruges og hvordan er den implementeret i projektet.

%We have had experience with different development methods from previous semesters, more specifically Scrum and XP.
%During these semesters we learned to take advantage of using common Scrum principles such as stand-up meetings, making short and precise iterations and to postpone tasks to the next iteration if not completed.
%We are, however, aware that Scrum might not fit exactly on this project.
%Therefore we have analyzed what is required from this project and how we need to modify Scrum in order to fit to our needs. \\

%We will implement all the principles from Scrum. In addition to this, we have changed and/or added the following to our modified version of Scrum:
%\begin{itemize}
%    \item Pair programming. We have decided to add the principles of pair programming, found in XP, to our project. We are working in an area of great uncertainty, as we have no prior experience with a lot of the problem domain. To lower the number of errors and faster get an understanding of our system, we have decided to implement Pair Programming.
%    \item Refactoring. Also due to the uncertainty of this project, we have added refactoring sessions to our iterations. These are also normally found in XP. They are added to enable us to improve our code over time.
%\end{itemize}

%The reason why we have chosen not to use any of the traditional development methods such as the Waterfall model is that its procedural and locked procedures does not fit a project, where the problem domain contains a lot of uncertainties and unknown areas. We need to be able to look back at what we have done and learned in the previous iterations, refactor, re-design, improve and re-implement parts of the solution. Our modified version of SCRUM allows us to accomplish this.



The choice of development method is based on previous experience and which methods suit the project.
The suited development method is determined by the circumstances under which the project is conducted. The project is conducted in a group of five members with a hard deadline.
All participants in the project have similar experience with software development methods, both agile and traditional methods.
The problem domain for the project is well defined, since there are no uncertainties in regards to the required functionality of the system.
The functionality defined by use cases does however contain unknown subjects such as video streaming, distributed computing, and communication with a drone.
These unknown subjects create an uncertainty about how to implement the desired functionality.
Another to consider is the participants motivation, which can be affected by the chosen development method.\\

From the circumstances it can be derived that a traditional development does not fit the project.
Traditional developments methods are suited for projects where the solution domain is known and it is possible to create a upfront design.
We deem this is not the case for this project, due to the uncertainty of the solution domain.
The uncertainty in the solution domain is based on the unknown subjects mentioned.
Furthermore the hard deadline means that with a traditional method there is the risk that very little, or even none of the functionality is implemented, as the design phase might become to long or complex due to the uncertainties in the solution domain.\\

Therefore an agile development method is better suited for this project.
An iterative development method allows for refactoring, redesign, and most importantly for this project, the possibility of not implementing some functionality to meet the hard deadline.
The agile development methods considered for this project are Extreme Programming \textit{(XP)} and Scrum \citep{Larman04}.\\

Both development methods have practices useful for the project.
Therefore the development process for the project is a combination of the two.
The practices taken from Scrum are Sprint planning meetings, A Sprint backlog of the tasks for the current spring, Daily stand-up meetings.
The practices taken from XP are Pair-programming and test-driven development.
The type of test used during test-driven development is acceptance test.