\section{Problem Domain}
\label{sec:problem_domain}
%Hvilke overvejelser skal der g�res omkring den wireless connection
%Hvilke overvejelser skal der g�res omkring at gemme videoen ned
%Hvilke overvejelser skal der g�res jurisik i forhold til overv�gning
%Hvilke overvejelser skal der g�res omkring sikkerhed i systemet og person f�lsom data
%Hvad er hovedpunkterne i problem domain
%Hvilke krav stiller de til l�sningen
The problem domain for the application is video surveillance of large outdoor areas.
The solution investigated in this report is to use a drone with a camera instead of stationary cameras to handle this.
The solution is intended to be used commercially by different businesses that may each have several different areas or locations that needs to be survelied.\\

Video surveillance using drones has a set of challenges which must be considered.
The connection to the drone must be stable at all times to make it controlable.
If the connection to the drone is lost the drone might crash, which must be avoided.
The large outdoor areas makes it a requirement that the connection is stable at long range.
Additionally the connection must support interaction with other services, making it possible to remotely control the drone.
Furthermore the system must be scalable to include several locations, meaning several drones, and users simultaneously. \\

For video surveillance to be useful the video must be stored.
It can be stored directly on the drone, or it can be stored externally.
Storing it on the drone increases the hardware requirements for the drone, as it must be ensured the data is not lost should the drone crash.
The alternative is to stream the video feed via the connection to the drone and store it externally.
This reduces the requirements to the drone, but increases the requirements to the connection as the video feed must be usable.\\

There are two legal aspects to consider with regards to video surveillance.
Firstly there are laws that limit the areas on which it is allowed to video surveil \citep{lov_om_tv_overvaagning} in Denmark, where this project is developed.
This is not an issue for video surveillance with stationary cameras, as they are simply placed in locations that ensure they do not violate these laws.
With movable cameras it is not trivial to ensure the video surveillance is done within the given laws. 
Furthermore the system must be secure.
In this context secure means that access to the sensitive information is restricted, both internally and externally\\

 
The main problems present in the problem domain are therefore as follows:\\
\begin{itemize}
	\item Wireless short-distance (with the drone, see Appendix~\ref{app:ar_drone_specification}) and wired long-distance communication (back to the user) with a drone. This includes controlling the drone.
	%We know that a drone will be near an area that it shall survival, and that the operator might be located far away. Therefore it is necessary that the system can communicate over long distances. 
	\item Wireless short-distance and wired long-distance video streaming. The drone must be able to send back the video feed from its cameras. Therefore the system must also be able to transmit a video feed in real-time over long distances.
	%\item Interaction with the drone will be handled through a web application. It is already decided that the user must be able to control the drone via a web application. Therefore we need to look into how one can do this, knowing that the web is a state-less environment. 
	\item Permissions- and access control in the web application. Includes both access to the system and access to specific drones.
	\item Scalability - the system must be scalable in terms of users and drones.
\end{itemize}

From these problems a set of requirements for \projectname{} can be derived:\\

\begin{itemize}
	\item A user must be able to login to the system.
	\item The system must have security measures in place to make sure any user only sees what he is permitted to.
	\item A drone must be able to send its video feed to the web service.
	\item A drone must be controllable from the web service.
	\item The system must support multiple drones and users.
\end{itemize}

These requirements are the base line for the use cases, which will be used to define the functionality of the system. 
