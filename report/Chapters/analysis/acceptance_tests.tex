\section{Acceptance Tests}
%Hvad er form�let med acceptance tests
%Hvordan er de linket til Use cases
%Link til design
The development method used in this project makes use of acceptance tests, see Section~\ref{sec:developmet_method}.
Acceptance tests are associated with use cases, and must be passed in order for a use case to have been implemented.
The acceptance tests are written based on the use cases and can be seen in Figure~\ref{tab:acceptance_tests1} and Figure~\ref{tab:acceptance_tests2}.
A use case may have several acceptance test associated with it.
In Figure~\ref{tab:acceptance_tests1} and Figure~\ref{tab:acceptance_tests2} the column \textbf{Use Case ID} refers to the use case the acceptance is associated with.
The use case IDs can be seen in Figure~\ref{tab:use_cases}.\\

\begin{figure}[htb]
\begin{center}
\begin{tabular}{ | l | l | p{8cm} | }
  \hline
	\textbf{ID} & \textbf{Use Case ID} & \textbf{Acceptance test} \\ \hline
	%%%%%%%%%%%      %%%%%%%%%%%S
	%%%%%%%%%%% OBS! %%%%%%%%%%%
	%%%%%%%%%%%      %%%%%%%%%%%
	% Inden tallene �ndres, skal der lige gøres opmærksom på det, da de bruges andre steder.
	1 & 1 & The user is provided with a username and password form, that gives visual feedback based on the users action (logging in, failed attempt).\\ \hline
	2 & 2 & The user, after performing a valid login, is only shown content according to his privileges. \\ \hline
	3 & 3 & The user with rights is shown a page with an interface that allows him to pilot a specific drone.\\ \hline
	4 & 4 & The user with rights is shown a page with a window that enables him to see the video feed of a specific drone. \\ \hline
	5 & 5 & The user can successfully grant or revoke a privilege to another user. \\ \hline
	6 & 6 & The user is able to change the name of the drone. \\ \hline
	7 & 7 & The user is able to link a drone to a company.\\ \hline
	8 & 7 & The user is able to unlink a drone from a company.  \\ \hline
	9 & 8 & The user is presented with a concise list of available drones.  \\ \hline
	10 & 9 & The user is able to press a link to logout of the system.  \\ \hline
	11 & 10 & The user with rights is able to create a new user via an interface.  \\ \hline
	12 & 10 & The user with rights is able to edit an existing user via an interface.  \\ \hline
	13 & 10 & The user with rights is able to deactivate an existing user via an interface.  \\ \hline
	14 & 10 & The user with rights is able to activate an existing user via an interface.  \\ \hline
	15 & 11 & The user is able to create a company via an interface.  \\ \hline
	16 & 11 & The user with rights is able to remove a company. \\
  \hline
\end{tabular}
\caption{Acceptance Tests 1}
\label{tab:acceptance_tests1}
\end{center}
\end{figure}

\begin{figure}[htb]
\begin{center}
\begin{tabular}{ | l | l | p{8cm} | }
  \hline
	\textbf{ID} & \textbf{Use Case ID} & \textbf{Acceptance test} \\ \hline
	%%%%%%%%%%%      %%%%%%%%%%%S
	%%%%%%%%%%% OBS! %%%%%%%%%%%
	%%%%%%%%%%%      %%%%%%%%%%%
	% Inden tallene �ndres, skal der lige gøres opmærksom på det, da de bruges andre steder.
	17 & 12 & The user is able to add users to the company.   \\ \hline
	18 & 12 & The user is able to remove users from the company.  \\ \hline
	19 & 12 & The user is able to add new roles to the company.  \\ \hline
	20 & 12 & The user is able to edit existing roles in the company.  \\ \hline
	21 & 12 & The user is able to remove existing roles from a company.  \\ \hline
	22 & 13 & The user with rights is able to grant his own privileges to another user within the same company.  \\ \hline
	23 & 13 & The user with rights is able to remove privileges from other users within the same company that he is able to grant them.  \\ \hline
	24 & 14 & The user is able to register as a user.  \\ \hline
	25 & 15 & The user with rights can add users to roles.  \\ \hline
	26 & 15 & The user with rights can remove users from roles.  \\
  \hline
\end{tabular}
\caption{Acceptance tests 2}
\label{tab:acceptance_tests2}
\end{center}
\end{figure}

