\section{Approach}
\label{sec:testing_approach}

There are a number of ways that \projectname{} can be tested:

\begin{itemize}
	\item Black-box testing
		\begin{itemize}
			\item Static -- Inspection and review of specifications i.e.
			\item Dynamic -- Acceptance tests.
		\end{itemize}

	\item White-box testing
		\begin{itemize}
			\item Static -- Code inspection and review
			\item Dynamic -- Unit-testing
		\end{itemize}
\end{itemize}

Inspection and review of the specifications and code inspection and review was done in every iteration, ensuring that \projectname{} was heading in the right directly accordingly to the requirements. 
This has been a process during the development, and the final results are what is presented in the design and implementation of \projectname{}.
Actual testing of \projectname{} has been limited to dynamic black-box testing.
We have chosen to only use dynamic black-box testing, since it is closely related to our previous approach using use cases and defining acceptance tests.
If the system were to go straight from development to production, however, unit-testing should be applied.
However, unit-testing has been neglected in this context. \\

A number of use-cases formed the basis of the design and implementation of \projectname{}.
These use-cases also formed the basis of the acceptance-tests that defines what we expect the software to do and how it should behave. \\

All of the acceptance tests were defined in Table~\ref{tab:acceptance_tests1} and Table~\ref{tab:acceptance_tests2} and they form the base of the dynamic black-box testing.
All of the tests are manually performed by two teams of two persons each.
The results are logged in Appendix~\ref{appendix:acceptance_test_results}. \\

%All of the tests are run two times. 
%After the first run, the results are evaluated and changes/improvements to \projectname{} are implemented. 
%Then a second acceptance test is run to ensure that the changes and improvements to \projectname{} has been correctly implemented according to %the use cases. 

% Alternativer:
% unit test
% blackbox er valgt > unit test fordi det kan laves inden programmeringen startes. Så har vi noget at gå efter.
