\section{Approach}
\label{sec:testing_approach}
There exist two testing approaches: Black-box testing and White-box testing, which each have a dynamic and static methods as listed below.\fxfatal{source til testing bog}

\begin{itemize}
	\item Black-box testing
		\begin{itemize}
			\item Static -- Inspection and review of specifications i.e.
			\item Dynamic -- Acceptance tests.
		\end{itemize}

	\item White-box testing
		\begin{itemize}
			\item Static -- Code inspection and review
			\item Dynamic -- Unit-testing
		\end{itemize}
\end{itemize}

Two testing approaches is used in \projectname{}, static White-box testing and dynamic Black-box testing.
Static White-box testing was conducted continuously throughout the development.
This was a result of pair programming being a part of the development method, as described in Section~\ref{sec:developmet_method}.
Pair programming is code review as the code is being written\fxfatal{source}.
Testing following completion of development has been limited to dynamic black-box testing.
We have chosen to only use dynamic black-box testing, since it is closely related to our previous approach using use cases and defining acceptance tests seen in Section~\ref{sec:acceptance_tests}, and as \projectname{} is only a proof of concept.\\

If \projectname{} should go from proof-of-concept to deployment more extensive testing is needed.
Unit-testing should be applied to ensure the functionality in \projectname{} is fit to use.
%Unit-testing provides automated tests that is often used to check whether or not the software being tested takes the correct input and gives the correct output. 
%This is to some degree also tested with acceptance tests, as different input is tested up against the system. 
%But with acceptance tests, this is made manually. 
%Unit tests are automatic and more complete than the acceptance tests as many different generated test inputs and outputs are tested. 
The two reasons we have chosen not to use unit-tests but reside to acceptance tests are that: 1) Setting up and executing unit tests is very time consuming and 2) \projectname{} is a proof of concept and therefore acceptance tests are deemed sufficient.\\

%A number of use-cases formed the basis of the design and implementation of \projectname{}.
%These use-cases also formed the basis of the acceptance-tests that defines what we expect the software to do and how it should behave. \\

The acceptance tests are defined in Figure~\ref{tab:acceptance_tests1} and Figure~\ref{tab:acceptance_tests2}.
They are derived from the use cases seen in Figure~\ref{tab:use_cases}.% and they form the base of the dynamic black-box testing.
All of the tests are manually performed by two teams of two persons each. \\

%All of the tests are run two times. 
%After the first run, the results are evaluated and changes/improvements to \projectname{} are implemented. 
%Then a second acceptance test is run to ensure that the changes and improvements to \projectname{} has been correctly implemented according to %the use cases. 

% Alternativer:
% unit test
% blackbox er valgt > unit test fordi det kan laves inden programmeringen startes. Så har vi noget at gå efter.
