\section{Problem Statement}
\label{sec:problem_definition}
Video surveillance in its current form is stationary, meaning cameras cannot move out of their mounted position.
As a consequence of this, the amount of hardware required is proportional to the size of the area to be surveilled.
%As a consequence of this, a large amount of hardware is required to properly surveil an area.
This means that a large area is expensive to surveil, due to the costs of hardware.
%This is problematic when a large area has to be surveilled, because the cost of hardware is high and the larger the area the more hardware is to be used.
Drone technology offers a possible solution to this problem by making video surveillance dynamic through movable cameras.
This possible solution yields the following preliminary problem statement:\\

\textit{How can drone technology be applied in a software application to improve the efficiency of video surveillance?}\\

From the preliminary problem statement the following aspects must be considered:
\begin{itemize}
	\item How to control a drones remotely over long distances.
	\item How to provide video streaming of a drone's camera through a web application.
	\item How to make a web application scalable to support multiple drones and users.
	\item How the make a system accessible from remote locations in a secure manner.
\end{itemize}

%For this project a AR Drone 2.0 Parrot, see Figure~\ref{fig:pic_of_drone}, has been acquired.

With the preliminary problem statement and the following aspects considered the following problem statement were yielded:: \\

\textit{How can drone technology be applied in a scalable web application to improve the efficiency and cost-efficiency of remote video surveillance of large outdoor areas?}