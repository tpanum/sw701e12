\section{Drone Technology}
As described in Section~\ref{sec:video_surveillance} the main problems with surveillance is the amount of hardware required, particularly cables, and cameras limited field of view.
A possible solution is to switch from fixed cameras to movable cameras.
A movable camera is one capable of moving freely in an environment under surveillance.
This can be achieved with the use of drones with an on board camera.
A drone in \projectname{} is an unmanned aircraft also called UAV which is remote controlled.
Drones capable of moving freely in an environment are capable of surveilling a larger area with fewer cameras than a traditional setup.
Drones ability to move around objects also reduce the problems with blind spots and limited field of view.
Surveilling an area with drones makes the surveillance responsive.
Responsive surveillance means an area is only surveilled when it is needed.
As an example an area might only be surveilled by a drone, when an alarm is triggered in the area i.e. the drone is not in the air at all times.
The need for cables can also be removed by using drones, as they can be controlled using wireless technologies such as WLAN \citep{ardrone_developer_guide}, radio waves or satellite \citep{drone_freq}.
In short a surveillance setup with drones requires less hardware than a setup with regular cameras, and surveillance can be made more responsive.