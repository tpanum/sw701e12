\section{Drone Technology}
As mentioned in Section~\ref{sec:video_surveillance} the main problems with surveillance is the ammount of hardware required, particularly cables, and the limited field of view of cameras.
A possible solution is to switch from fixed cameras to movable cameras.
A movable camera is one capable of moving freely in the environment it is to surveil.
This can be achieved with the use of drones with an onboard camera.
A drone in this context is a pilotless aircraft controlled via remote control.
Drones capable of moving freely in an environment are capable of surveilling a larger area with fewer cameras, than a traditional setup.
Their ability to move arround objects also reduce the problems with blind spots.
Furthermore their ability to move can make surveillance responsive.
In this context responsive surveillance means an area is only surveilled when it is needed.
As an example an area might only be surveilled by a drone, when an alarm is triggered in the area.
The need for cables is also removed by using drones, as they are typically controlled using wireless internet technologies or radio waves\fixme{source}.
In short a surveillance setup with drones require less hardware than a setup with regular cameras, and surveillance can be made more responsive.


