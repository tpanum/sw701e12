
\section{Future work}
A number of the acceptance tests failed.
The tests which were not passed must be implemented and bugs must be corrected before \projectname{} is ready for deployment \\

Dynamic black-box testing, through acceptance tests, was the only testing performed after development ended on \projectname{}.
To move from the current proof of concept state to on ready for deployment, \projectname{} must be subjected to more extensive testing.
This testing should include both unit testing to eliminate bugs, and usability testing.

The streaming solution made for \projectname{} is not ideal from an use case perspective.
To implement a solution usable in \projectname{} would require developing a streaming tool, capable of parsing PaVE headers and forward a video stream displayable in a Flash application. \\

Those are the issues that must be resolved in the project's current state
There are however a set of additional items that must be addressed to make \projectname{} ready for deployment:


\begin{itemize}
	\item Radio-controlled drones -- It is not an optimal solution that the drones broadcast their own WIFI network that \deno{S} must connect to in order to enable communication with the drone. An alternative solution is communicate with the drone through radio signals, which gives an advantage in terms operational range of the drone. The maximum range of the Ar Drone's Wi-Fi signal is 50 meters \citep{wifirange}, where as a radio controlled drones can have operational lengths of up to 200 kilometers \citep{uavrange}.  
	\item Better hardware -- The current drones has limited fly time due to the batteries. The Wi-Fi network card used in \deno{S} is of low quality. It has a limited range, and its driver contains a bug which causes \deno{S} to crash whenever the connection to the drone is lost. This hardware needs to be improved before \projectname{} deployed.
	\item Recording of video stream - The video stream must be recorded on \deno{S} to ensure video documentation is available.
\end{itemize}

%Once this list of fixes and improvements has been implemented, \projectname{} is getting close to a ready to market state. After completing an intense testing session, of course. 
