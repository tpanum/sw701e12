
% streaming - det virker ikke som vi ville, hvad gjorde vi / hvad ville vi gerne have gjort
% security på slaven - ukrypteret nøgle sendes - se stream selvom du ikke har adgang

% use case perspektiv er steaming ikke opfyldt, men vi kan godt gøre "sådan her..!"

% godkendelse af nye slaver - de har et unikt id men alle kan sådan set godkende sig as is now. Mangler db på master med kendte slaver

% det har kostet os noget tid at lære Rails fremfor at bruge noget vi kender.




% hvor godt er det vi har lavet ift. det vi ville?
% rails fremfor php/noget vi er kendt i?
% forhold os til de valg vi har truffet
% vær kritisk overfor det vi har valgt her.


% STRUKTUR
% Done! Rails - god/dårlig beslutning? Hvad har det tilført projektet?
% Done! security problem - det er ukrypteret trafik og man kan spoof nye slaver.
% Done! streaming - hvorfor virker vores idé ikke? Hvad kunne man have gjort? Hvad har vi så gjort i stedet? Fordele/ulemper. 
% Done! Arbejdsblade - tidsforbrug, godt/skidt? 

% Reflektér over det endelige produkt



\section{Discussion \& Reflection}
\label{sec:discussion}
A series of decisions were made, whcih had an affect on \projectname{}.
This section will elaborate on these decisions and the affect they had. \\

A decision was made to use the programming language Ruby and the framework \ac{Rails}.
Rails contains functionality which made the development of \projectname{} easier such as native support for the MVC model and Active records.
This functionality optimized the time spend on development.
Furthermore this is a student project, and Ruby and Rails were new technologies to us, that we wanted to learn.
As a result time had to be dedicated to learning these new technologies.
This resulted in a trade-off where the time saved by using Ruby and Rails had to spend on learning the language and the framework.
Had different technologies been used, which we have experience with, 
We may have been able to implement this project faster if we were using a programming language that we were more familiar with, such as PHP, but we believe that the learning process is worth more than the time we might have saved. \\

Looking at the final product, \projectname{}, we are satisfied with the results. 
There are, however, elements in it that could be improved. 
Some are important to look at while others are less important until the product should be made ready for market. \\

Whenever a client (Browser \deno{B}) is trying to access a Drone \deno{D}, it must receive a session key from \deno{M} generated on \deno{S}.
This key is transfered unencrypted via regular HTTP. 
This means that the key can be caught by third party if he can sniff the network packages sent on the network of either \deno{S} or \deno{B}. 
This has no impact on the functionality of the system, but it is a security concern that should be looked at before \projectname{} is ready to market.

Another potential security issue is the registration of new Slaves \deno{S}.
There is no authentication of new \deno{S} in the current implementation, meaning that any computer can connect to \deno{M}.
If they provide a valid serial ID when they connect the first time, they will be saved in the system as an active and valid \deno{S}. 
This can lead to different security issues, but as the previous security issue it has no significant impact no the functionality of \projectname{}.
It must, however, be fixed before \projectname{} can be seen as ready to market. \\

A lot of our time was spent on streaming the video feed from the drone to the clients browser. 
Transporting the video feed from the drone to the slave \deno{S} was no problem, for this purpose Gstreamer was used and set up with appropriate options on \deno{S} to fetch the feed from the drone. 
Displaying the video feed in the flash application in the clients browser did not work because of the lack of PaVE headers as described in Section~\ref{sec:issues_with_pave_headers}.
A lot of time was spent on this, as displaying the video feed from the drone is one of the important features of \projectname{}. 
Instead we ended up implementing the solution described in Section~\ref{sec:video_stream_implemented_solution}, which works (sort of) but in no way is an ideal implementation.
It requires the client to run a separate application on his computer to view the video stream, and have his computer focused on the flash application in the browser at the same time to be able to send commands to the drone. 

This is not a satisfying solution.
However, it works as a proof of concept implementation. 
It is possible to both view the video feed from the drone and send commands to it that it will react too. 
If \projectname{} was suppose to be ready to market now, a solution where the video feed is presented in the browser, preferably in the same flash application that takes the flight commands, is a requirement. 

A lot of time was spent trying to implement the first solution, as we are aware of the importance if this, given the case that \projectname{} should be ready to market. 
However, when we came to the conclusion that we could not get it to work within the given time of this project, the secondary solution was implemented, which -- in our eyes -- is better than no video feed and still makes a proof of concept. 
We do not consider the time wasted, as it still provided the project with a lot of valuable knowledge about streaming and how to implement such solutions. \\

During the design and implementation of \projectname{} we have used work sheets to document our thoughts and work, to make it easier when we had to write this report. 
This also took a lot of time, but is considered a good investment as it provided useful resources in writing this report. \\

\projectname{} is not a perfect system, and it is not ready to market.
However, there are elements in it that we are very proud of, and that we think is good work. 

\begin{itemize}
	\item The architecture that binds a drone and a user together, using a browser, slave and master-server. 
	\item The object-model of the structure of the web application hosted on the master.
	\item The communication with the drone that allows us to control its movements via a flash application in the clients browser.
\end{itemize}

These are the most basic elements in \projectname{}\'s foundation, that we are proud of and believe are well implemented.






