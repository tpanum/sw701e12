
% streaming - det virker ikke som vi ville, hvad gjorde vi / hvad ville vi gerne have gjort
% security på slaven - ukrypteret nøgle sendes - se stream selvom du ikke har adgang

% use case perspektiv er steaming ikke opfyldt, men vi kan godt gøre "sådan her..!"

% godkendelse af nye slaver - de har et unikt id men alle kan sådan set godkende sig as is now. Mangler db på master med kendte slaver

% det har kostet os noget tid at lære Rails fremfor at bruge noget vi kender.




% hvor godt er det vi har lavet ift. det vi ville?
% rails fremfor php/noget vi er kendt i?
% forhold os til de valg vi har truffet
% vær kritisk overfor det vi har valgt her.


% STRUKTUR
% Done! Rails - god/dårlig beslutning? Hvad har det tilført projektet?
% Done! security problem - det er ukrypteret trafik og man kan spoof nye slaver.
% Done! streaming - hvorfor virker vores idé ikke? Hvad kunne man have gjort? Hvad har vi så gjort i stedet? Fordele/ulemper.
% Done! Arbejdsblade - tidsforbrug, godt/skidt?

% Reflektér over det endelige produkt



\section{Discussion}
\label{sec:discussion}
A series of design choices were made throughout the development phase of \projectname{}.
Some of these will be discussed in this section. \\

One decision was made to use the programming language Ruby and the framework \ac{Rails}.
\acs{Rails} contains functionality, which made the development of \projectname{} easier, such as support for the \ac{MVC} design pattern and Active Records.
This functionality optimized the time spend on development.
This is a student project, which means no personal resources are involved, except already preallocated time.
This enables us to focus on learning new languages with less concerns about meeting the deadline with the granted resources.
This meant that some of our time had to be dedicated to learning these new technologies. % As a result time had to be dedicated to learning these new technologies.
Learning a new language involves a trade-off of what the new language provides, in terms of time saved, opposed to the time spent learning it. %This resulted in a trade-off where the time saved by using Ruby and Rails had to spend on learning the language and the framework.
Had we used technologies in which the whole group had experience, we may have been able to implement the project faster.
Such a technology could be PHP, which the whole group has experience with.
We decided that the learning process would be worth more than the time we might have saved. \\
% Had different technologies been used, which we have experience with,
% We may have been able to implement this project faster if we were using a programming language that we were more familiar with, such as PHP, but we believe that the learning process is worth more than the time we might have saved. \\

% Looking at the final product, \projectname{}, we are satisfied with the results.
There are some unresolved issues in \projectname{}.
These issues can be prioritized according to their need to be resolved in order to have deployable solution. \\
% Some of these is are important to look at while others are less important until the product should be made ready for market. \\

Whenever a user (Browser \deno{B}) is trying to access a drone, it must receive a session key from the web application on Master \deno{M} generated on the Slave \deno{S} associated with the drone.
This key is transferred unencrypted over regular HTTP.
This means that the key can be read by viewing packages sent on the network, since it is sent as raw data.
This has no impact on the functionality of the system, but it is a security concern that should be addressed before \projectname{} is ready to be deployed.

Another potential security issue is the registration of new \deno{S}.
There is no authentication of new \deno{S} in the current implementation, meaning that any computer can connect to \deno{M}.
If they provide a valid serial ID when they connect the first time, they will be saved in the system as an active and valid \deno{S}.
These security issues does not affect the functionality of \projectname{}.
% This can lead to different security issues, but as the previous security issue it has no significant impact no the functionality of \projectname{}.
These must be resolved before \projectname{} can be deployed. \\

Some of our time was spent on streaming the video stream from the drone to \deno{B}.
The amount of time was more than expected.
GStreamer was used and set up with appropriate options on \deno{S} to fetch the stream from the drone.
It was not possible to display the video stream in the Flash application on B due to issues with the PaVE headers as described in more detail in Section~\ref{sec:issues_with_pave_headers}.
Some of our time was spent on this, as displaying the video stream from the drone is one of the important features of \projectname{}.
This amount of time was more than expected.
Instead we implemented an alternative solution, which did not meet the requirements of \projectname{}.
This solution is described in more detail in Section~\ref{sec:video_stream_implemented_solution}.
It requires the user to run a separate application to view the video stream, and have the Flash application focused in the browser at the same time to be able to send commands to the drone.

This solution is fulfilling as a proof of concept.
It is possible to both view the video stream from the drone and send commands to it.
If \projectname{} was supposed to be deployed, a solution where the video stream is presented in the browser, in the same Flash application that sends the commands, is a requirement. \\

The time used for implementing the design exceeded our expectations, and was due implementation issues of the streaming.
However, since the implementation does not include the design of streaming.
This is an issue that has to be resolved before \projectname{} is able to be deployed.

% Some of our time was spent trying to implement the first solution, as we are aware of the importance if this, given the case that \projectname{} should be ready to market.
% The amount of time was more than expected.
% However, when we came to the conclusion that we could not get it to work within the given time frame of this project, the secondary solution was implemented, which -- in our eyes -- is better than no video stream and still makes a proof of concept.
% We do not consider the time wasted, as it still provided the project with a lot of valuable knowledge about streaming and how to implement such solutions. \\

During the design and implementation of \projectname{} we used worksheets to document our thoughts and work.
This enabled us to write down design decisions throughout the development process.
This allowed us to remember the design decisions during the writing of the report. \\

% \projectname{} is not a perfect system, and it is not ready to market.
% However, there are elements in it that we are very proud of, and that we think is good work.

% We deem that the following design choices were well decided:

% The architecture behind LONE t

% \begin{itemize}
% 	\item The architecture that binds a drone and a user together, using a browser, slave and master-server.
% 	\item The object-model of the structure of the web application hosted on the master.
% 	\item The communication with the drone that allows us to control its movements via a Flash application on \deno{B}.
% \end{itemize}






